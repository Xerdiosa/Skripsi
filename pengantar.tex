%-----------------------------------------------------------------------------%
\chapter*{\kataPengantar}
%-----------------------------------------------------------------------------%

Puji syukur penulis panjatkan kepada Allah SWT karena berkat rahmat dan nikmat-
Nya penulis dapat menyelesaikan tugas akhir dengan judul “\judul” ini. Penyusunan tugas akhir ini dilakukan dalam rangka memenuhi
salah satu persyaratan untuk memperoleh gelar Sarjana Ilmu Komputer. Dalam
jalannya proses penulisan tugas akhir ini, penulis mendapatkan banyak bantuan,
bimbingan, arahan, dan serta dukungan dari berbagai pihak. Untuk ini, penulis ingin
menyampaikan rasa terima kasih yang sebesar-besarnya kepada:
\begin{enumerate}[topsep=0pt,itemsep=-1ex,partopsep=1ex,parsep=1ex]
    \item Orang tua penulis, Drs. Endri Rizal dan Imelda Himawati, SE. Ak. yang selalu memberi dukungan agar penulis dapat mengikuti studi hingga saat ini.
    \item Prof. DR. H. Amir Luthfi(datuk), alm. Amir Hamzah(opa), Dra. Hj. Asmah Salut(nenek), dan almh. Hj. Emmy(oma) selaku kakek dan nenek penulis yang selalu mendukung penulis dari kecil hingga sampai saat ini.
    \item Nadya Rahma Zafira, SH. dan Galuh Widyatama Nugraha, S.TP, kakak dan abang ipar penulis yang terus menyemangati studi penulis dari saat masih maba hingga sekarang ini.
	\item Bapak Gladhi Guarddin, S.Kom., M.Kom. sebagai pembimbing Tugas Akhir penulis yang telah membantu dari segi materi, penulisan, dan moral dengan sangat baik dan sabar sehingga Tugas Akhir ini selesai.
	\item Bapak Adila Alfa Krisnadhi, S.Kom., M.Sc., Ph.D. sebagai pembimbing akademis penulis yang telah membimbing penulis selama 4 tahun masa perkuliahan penulis.
	\item Collyn Power, S.Kom., selaku mentor penulis dari pengerjaan Tugas Akhir yang telah membimbing selama proses pengembangan aplikasi Tugas Akhir ini.
	\item PT Gudang Ada Globalindo selaku tempat penulis melakukan penelitan ini.
	\item Teman-teman \textit{Not Safe for Plebs} (NSFP), Nadhif AP, Rocky, Krisna, Timothy, Adrian, Galang, Egi, dan Nadhif S, sirkel penulis yang selalu melalui bersama-sama melalui tantangan di Fasilkom bersama-sama sejak tahun pertama.
% 	\item Hololive Indonesia, terutama Kobo Kanaeru yang telah menemani penulis selama pengerjaan tugas akhir ini. ~ehe
% 	\item Seluruh pihak yang tidak dapat penulis sebutkan satu persatu, sehingga penulis bisa sampai di titik ini.
\end{enumerate}

Penulis menyadari bahwa laporan \type~ini masih jauh dari sempurna. Oleh karena itu, apabila terdapat kesalahan atau kekurangan dalam laporan ini, Penulis memohon agar kritik dan saran bisa disampaikan langsung melalui \f{e-mail} \code{aulia.aa.akbar@gmail.com}.

\vspace*{0.1cm}
\begin{flushright}
Depok, \tanggalSiapSidang\\[0.1cm]
\vspace*{1cm}
\penulis

\end{flushright}
