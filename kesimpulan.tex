%---------------------------------------------------------------
\chapter{\kesimpulan}
\label{bab:6}
%---------------------------------------------------------------
Pada bab ini, akan dipaparkan kesimpulan penelitian dan saran untuk penelitian berikut-nya.


%---------------------------------------------------------------
\section{Kesimpulan}
\label{sec:kesimpulan}
%---------------------------------------------------------------
Berikut ini adalah kesimpulan terkait pekerjaan yang dilakukan dalam penelitian ini:
\begin{enumerate}
	\item \bo{Integrasi Helm} \\
	Proses \textit{deployment} helm dapat diintegrasikan dengan aplikasi yang bekerja melalui \textit{HTTP request} dengan mengimplementasikan aplikasi yang memakai Helm \textit{API}. Aplikasi tersebut akan diimplementasikan dengan bahasa pemrograman Golang. Aplikasi akan menerima \textit{HTTP request} dari pengguna, setelah itu akan melakukan \textit{parsing data}, melakukan \textit{preprocessing}, dan terakhir akan meneruskannya ke API milik Helm untuk di \textit{deploy}.
	\item \bo{Integrasi Vault dan Amazon Kinesis} \\
	Dari rancangan pada bagian \ref{sec:vaultEngine} untuk Vault dan bagian \ref{sec:perancanganArsitektur} untuk Amazon Kinesis, serta implementasi pada bagian \ref{sec:initVault} dan \ref{sec:initKinesis}, dan telah diujikan fitur-fiturnya pada bab \ref{bab:5}. Dapat disimpulkan bahwa Vault dan Amazon Kinesis dapat diintegrasikan dengan aplikasi yang dikerjakan untuk membantu proses \textit{deployment}.
	\item \bo{Analisis terhadap Terraform} \\
    Pada bagian \ref{sec:terraform} dibuat spesifikasi pada Terraform dan dilakukan \textit{deployment} pada bagian \ref{sec:mergeTerraform}, proses tersebut membutuhkan pengguna untuk mendefinisikan 4 berkas untuk membuat satu buah \textit{deployment}. Sedangkan pada bagian \ref{sec:templateTest} hanya dibutuhkan 1 \textit{request} untuk membuat \textit{template} dan 1 \textit{request} untuk melakukan \textit{deployment}. Jika terdapat 5 \textit{deployment} dengan spesifikasi yang mirip, Terraform membutuhkan pengguna untuk menspesifikasikan 20 berkas, namun aplikasi yang dikembangkan hanya membutuhkan 6 \textit{HTTP request} untuk melakukan \textit{deployment} yang sama. Sehingga dapat disimpulkan bahwa penggunaan \textit{templating} pada aplikasi dapat mensimplifikasi proses \textit{deployment} dibandingkan dengan Terraform.
\end{enumerate}

%---------------------------------------------------------------
\section{Saran}
\label{sec:saran}
%---------------------------------------------------------------
Berdasarkan hasil penelitian ini, berikut ini adalah saran untuk pengembangan penelitian berikutnya:
\begin{enumerate}
	\item Kontrol aplikasi hanya dapat dilakukan pada Kubernetes Cluster yang sama pada tempat aplikasi tersebut berjalan. Hal ini dapat memaksa untuk dibuatnya banyak \textit{instance} aplikasi pada tiap Kubernetes Cluster. Kontrol antar cluster diharapkan untuk dapat diimplementasi untuk dapat menangani hal tersebut.
	\item Belum adanya sistem autentikasi dalam penggunaan aplikasi. Hal ini dapat menyebabkan siapapun yang dapat melakukan \textit{HTTP Request} ke aplikasi memiliki akses penuh kedalam aplikasi dan semua deployment yang dikontrol. Sistem autentikasi diharapkan untuk dapat diimplementasi untuk dapat menangani hal tersebut.
	\item Belum adanya \textit{Graphical User Interface}(GUI). Pengguna harus membuat HTTP Request melalui aplikasi pihak ketiga seperti CURL maupun Postman untuk dapat mengakses aplikasi. \textit{Website Front-end }diharapkan untuk dapat diimplementasi untuk dapat menangani hal tersebut.
\end{enumerate}
